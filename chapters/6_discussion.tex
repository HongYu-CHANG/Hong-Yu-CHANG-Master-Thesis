\chapter{Discussion} \label{chapter:discussion}

\section{Limitations OK}
FacePush has some limitations to overcome. We summarize these into three categories. First, after installing FacePush on the HMD, the weight of the whole system is easily perceived by the user. Most of the weight is caused by the motors on both sides, which may cause the HMD to slide down during the experience. Second, the motor generates some noise which can impact the user's experience. These limitations can overcome by using different actuation mechanisms, such as embedding Shape Memory Alloys into the belt, though whether that the mechanism allows sufficient speeds and strengths of forces suggested by the FacePush study requires further research. Finally, we use a pressure unit, which is proportional to the force exerted on the user's face, to represent the force in this work. Pressure is used due to the difference of facial characteristics (muscle density, etc) of the different users. In our future studies, we will address this by analyzing the different characteristics of the facial properties and facial interfaces of HMDs to quantify the actual forces.

\section{Future Work OK}
Currently, the haptic feedback of FacePush is focused only on the facial region of the user. In the future, we will apply FacePush's mechanism to other parts of the body, similar to HapticClench does on the wrist \cite{HapticClench} and GravityGrabber on the finger \cite{Gravity.Grabber}. In addition, we hope to combine FacePush's generated haptic feedback with other simulated facial feedback, such as tactile touch and thermal feedback.
