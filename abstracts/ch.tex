\newpage
\begin{center}
  \LARGE
  \begin{singlespace}
    \textbf{\titleCh} \\[0.5cm]
  \end{singlespace}

  \begin{singlespace}    
  \begin{tabular}{r l}
    研究生: & \studentCh \\
    指導教授: & \advisorCh \hspace{0.1cm} 教授 \\[0.5cm]
  \end{tabular}
  \end{singlespace}

  \universityCh \\
  \instituteCh  \\[0.5cm]
    
  \makebox[4em][s]{摘要} \\[0.5cm]
\end{center}

\normalsize 
本文介紹了FacePush,一種與頭戴式顯示器(HMD)組合的系統,可在虛擬現實(VR)中為使用者的臉部產生正向力。FacePush的機制是透過兩個馬達提供的扭力來產生的,這兩個馬達利用\textcolor{red}{滑輪系統}壓在使用者的臉上。FacePush可以產生不同強度的正向力,並將其應用於臉上。\\

為了讓FacePush在VR應用中獲得明顯和可辨別的正向力,我們進行了兩項研究,以確定絕對檢測閾值和使用者感知的識別閾值。在進一步考慮使用者舒適度之後,我們確定兩個水平的力,2.7 kPa和3.375 kPa,是透過實施三種應用來展示FacePush體驗的理想選擇,這三種應用證明了使用離散和連續正向力來實現虛擬現實中的拳擊,潛水和360導引。此外,關於拳擊的應用,我們進行了使用者研究,在享受度和現實感方面進行使用者體驗的評估並收集使用者的回饋。\\[0.7cm]

關鍵字:虛擬實境, 正向力, 臉部觸覺, 頭戴式顯示器